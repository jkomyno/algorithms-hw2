\section{Conclusioni}
\label{cap:conclusions}

In questa relazione abbiamo descritto gli algoritmi e relative le
scelte di implementazione, tempi di esecuzione, l'errore di
approssimazione e ottimizzazioni utilizzate per rendere i programmi
più efficienti. \\

\noindent Nella sezione \ref{cap:performance-analysis} abbiamo
risposto alle 2 principali domande dell'homework, mentre nella sezione
\ref{cap:benchmark-process} abbiamo descritto il processo di benchmark
adottato, pensato per essere quanto più affidabile e stabile
possibile.  Nelle sezioni \ref{cap:code-structure},
\ref{cap:implementation-choices} e \ref{cap:algorithms} abbiamo invece
discusso i dettagli tecnici, quali struttura, scelte implementative e
codice degli algoritmi. \\

\noindent La sezione \ref{cap:extensions-and-originalities} descrive
infine le estensioni esplorate per soddisfare la nostra curiosità e
arricchire il nostro bagaglio accademico, e permettendoci di esplorare
alternative anche migliori. \\

\noindent Questo progetto ci ha permesso di sperimentare diversi
algoritmi di approssimazione ed euristiche per un problema non banale
il cui algoritmo deterministico ha complessità non
polinomiale. L'impossibilità di calcolare una soluzione esatta anche
per dimesioni medio/grandi dell'input ci ha spinto a cercare algoritmi
nuovi e migliori rispetto ai tre scelti inizialmente per
l'homework. Abbiamo infatti sperimentato nuove euristiche come
FarthestInsertion e la sua implementazione alternativa
FarthestInsertionAlternative; abbiamo implementato SimulatedAnnealing,
che risolve il problema in un modo completamente diverso e escogitato
un modo di ottenere approssimazione ancora migliori semplicemente
sfruttando gli stessi algoritmi eseguiti con inizializzazione
differente.\\

\noindent Il progetto è disponibile anche come repository pubblica su Github:

\begin{center}
\href{https://github.com/jkomyno/algorithms-hw2}{github.com/jkomyno/algorithms-hw2}
\end{center}
