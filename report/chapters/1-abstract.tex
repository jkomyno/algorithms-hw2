\section{Abstract}
\label{cap:abstract}

Questo secondo homework di laboratorio di Algoritmi Avanzati ha lo scopo di implementare e confrontare algoritmi per risolvere il Traveling Salesman Problem nel caso metrico. \\

\noindent Gli algoritmi principali implementati sono tre:

\begin{enumerate}
    \item Algoritmo 2-approssimato per Metric-TSP basato sul Minimum Spanning Tree. In questo caso per il calcolo dell'MST abbiamo usato l'algoritmo di Prim;
    \item Algoritmo esatto di Held \& Karp, che usa la programmazione dinamica;
    \item Algoritmo che usa l'euristica costruttiva Closest Insertion per Metric-TSP.
\end{enumerate}

\noindent Abbiamo considerato anche alcuni contributi originali rispetto agli algoritmi visti in classe; esse sono discussi e presentati nella sezione \hyperref[cap:extensions-and-originalities]{Estensioni e originalità}. \\

\noindent Il codice è scritto in C++17 ed è opportunamente commentato per facilitarne la comprensione. Non è stata usata alcuna libreria esterna.

\noindent Le risposte alle 2 domande principali dell'homework sono riportate nella sezione \hyperref[cap:performance-analysis]{Analisi dei risultati}.
