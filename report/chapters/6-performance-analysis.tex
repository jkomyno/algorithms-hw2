\section{Analisi delle performance}
\label{cap:performance-analysis}

\subsection{Domanda \#1}

\begin{displayquote}
Eseguite i tre algoritmi che avete implementato (Held-Karp, euristica costruttiva e 2-approssimato) sui 13 grafi del dataset. Mostrate i risultati che avete ottenuto in una tabella come quella sottostante. Le righe della tabella corrispondono alle istanze del problema. Le colonne mostrano, per ogni algoritmo, il peso della soluzione trovata, il tempo di esecuzione e l'errore relativo calcolato come $(SoluzioneTrovata-SoluzioneOttima)/SoluzioneOttima$. Potete aggiungere altra informazione alla tabella che ritenete interessanti.
\end{displayquote}

\noindent --- \\


\subsection{Domanda \#2}

\begin{displayquote}
Commentate i risultati che avete ottenuto: come si comportano gli algoritmi rispetti alle varie istanze? C'è un algoritmo che riesce sempre a fare meglio degli altri rispetto all'errore di approssimazione? Quale dei tre algoritmi che avete implementato è più efficiente?
\end{displayquote}


\noindent ---\\
