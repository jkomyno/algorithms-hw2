
\section{Algoritmi}
\label{cap:algorithms}

\subsection{TSP 2-approssimato con Primm}

La versione vista a lezione di questo algoritmo prevede i seguenti step:

\begin{enumerate}
    \item Ricavare l'MST del grafo in input utilizzando l'algortimo di Primm e la radice da cui partire.
    \item Eseguire una visita pre-order dell'MST ricavato al passo precedente
    \item Aggiungere alla visita pre-order la radice.
    \item Calcolare il peso totale del circuito ricavato nei 2 passi precedente e restituire il risultato.
\end{enumerate}

\noindent Il listato \ref{listing:tsp2approx} contiene la nostra implementazione dell'algoritmo, step per step.\\

\begin{listing}[!ht]
\begin{minted}{c++}
    // Step: 1
    auto mst(mst::prim_binary_heap_mst(distance_matrix));

    // Step: 2-3
    DFS dfs(std::move(mst));
    const auto circuit = dfs.preorder_traversal_rec();

    const auto get_distance = [&distance_matrix](const size_t x, const size_t y) {
        return distance_matrix.at(x, y);
    };

    // Step: 4
    return utils::sum_weights_in_circuit(circuit.cbegin(), circuit.cend(), get_distance);

\end{minted}
\caption{Implementazione di TSP 2-approssimato. I commenti del file originale sono stati omessi per una maggiore compattezza.}
\label{listing:tsp2approx}
\end{listing}

\noindent L'algoritmo TSP 2-approssimato è stato implementato a partire dallo pseudo codice visto in classe. \\

\subsubsection{Osservazioni}

\begin{itemize}
    \item E' opportuno in questo caso utilizzare una matrice delle distanze (o matrice di adiacenza) in quanto essendo questi grafi completi l'implementazione di Primm diventa più efficiente e performante.
    \item La cosa di priorità per Primm è stata implementata con una Binary Heap.
\end{itemize}


\subsection{Held e Karp}

Riportiamo qui di seguito nel listato TODO lo pseudocodice dell'algortimo di Held e Karp:
TODO immagine?

\noindent Il listato \ref{listing:held-karp} contiene la nostra implementazione dell'algoritmo, step per step.

\begin{listing}[!ht]
\begin{minted}{c++}
// ----

\end{minted}
\caption{Implementazione di Held e Karp. I commenti del file originale sono stati omessi per una maggiore compattezza.}
\label{listing:held-karp}
\end{listing}

\noindent L'algoritmo di Held e Karp è stato implementato a partire dallo pseudo codice visto in classe. \\

\subsubsection{Osservazioni}

\begin{itemize}
    \item ----\\

    \item ----\\

\end{itemize}


\subsection{Farthest Insertion}

La versione vista a lezione di Farthest Insertion prevede i seguenti step:

\begin{enumerate}
    \item ---
    \item ---
\end{enumerate}

\noindent Il listato \ref{listing:farthest-insertion} contiene la nostra implementazione dell'algoritmo, step per step.

\begin{listing}[!ht]
\begin{minted}{c++}
// ---.h

\end{minted}
\caption{Implementazione di Farthest Insertion. I commenti del file originale sono stati omessi per una maggiore compattezza.}
\label{listing:farthest-insertion}
\end{listing}

\subsubsection{Osservazioni}
\begin{itemize}
    \item ----\\

\end{itemize}
