\section{Struttura del codice}
\label{cap:code-structure}

Il progetto è strutturato in un'unica soluzione Visual Studio\footnote{Una soluzione Visual Studio può essere vista come un macro-progetto che contiene più sotto-moduli.} contenente molteplici progetti, uno per ogni algoritmo per il Metric-TSP implementato. Il codice di ogni progetto è contenuto nella omonima cartella. Di seguito l'elenco dei progetti realizzati:

\begin{itemize}
    \item \textbf{MST2Approximation}: Implementazione dell'algoritmo di $2$-approssimazione basato sul Minimum Spanning Tree;
    \item \textbf{HeldKarp}: Implementazione dell'algoritmo esatto Held \& Karp con timeout di esecuzione fissato a 2 minuti;
    \item \textbf{ClosestInsertion}: Implementazione dell'algoritmo di $2$-approssimazione basato sull'euristica costruttiva Closest Insertion;
    \item $(*)$ \textbf{FarthestInsertion}: Implementazione dell'algoritmo di $log(n)$-approssimazione basato sull'euristica costruttiva Farthest Insertion;
    \item $(*)$ \textbf{SimulatedAnnealing}: Implementazione originale dell'algoritmo stocastico Simulated Annealing applicato al problema Metric-TSP, che utilizza l'euristica Nearest-Neighbor per generare la soluzione di partenza.
\end{itemize}

\noindent Abbiamo deciso di scegliere \textbf{Farthest Insertion} tra le euristiche costruttive elencate nell'homework.

\noindent I progetti indicati con $(*)$ sono delle estensioni o delle aggiunte rispetto ai 3 algoritmi inizialmente richiesti.
\\

\noindent La cartella \textit{Shared} contiene le strutture dati custom e alcune classi e metodi di utilità usati
condivisi tra progetti.

\noindent Abbiamo configurato Visual Studio per importare automaticamente i file di header salvati nella cartella \textit{Shared}
durante la compilazione di ogni sottoprogetto. Analogamente, tale cartella è inclusa nella compilazione del \textit{Makefile}, grazie all'opzione \textit{-I} del compilatore \textit{g++}.

