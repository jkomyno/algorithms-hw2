\section{Estensioni e originalità}
\label{cap:extensions-and-originalities}

Oltre alle tre implementazioni richieste dalla consegna dell'homework,
abbiamo deciso di esplorare qualche altra estensione degli algoritmi
per il problema del commesso viaggiatore.

\subsection{Closest Insertion}
\label{sec:closest-insertion}

\emph{Closest Insertion} è una delle euristiche impiegabili per la
risoluzione problema del TSP in modo approssimato. Questa euristica
è molto simile a \emph{Farthest Insertion} (si veda \ref{sec:farthest-insertion}),
infatti differisce da essa solo per la selezione del vertice $k$
da inserire nel circuito parziale $C$.

In particolare, in questo caso viene scelto il vertice $k$ non presente
nel circuito $C$ che minimizza $\delta (k, C) = \min_{h \in C} w(h, k)$.
Sostanzialmente viene scelto il vertice $k$ più vicino al ciclo $C$,
ma che non è ancora incluso in esso.

\subsubsection{Implementazione}

Il codice è sostanzialmente simile a quello riportato nel listato
\ref{listing:farthest-insertion}, con la differenza che per
\emph{Closest Insertion} si effettua la scelta del nuovo nodo
con \mintinline{c++}{utils::select_new_k} passandogli come quarto
argomento un comparatore che permette, in questo caso, di
minimizzare $\delta (k, C)$.

\begin{listing}[!ht]
\begin{minted}{c++}
// the comparator will be passed to std::max_element.
const auto min_comparator = [](const auto& x, const auto& y) {
    return x.second > y.second;
};
\end{minted}
\caption{Differenza di implementazione per Closest Insertion rispetto a Farthest Insertion.}
\label{listing:closest-insertion-diff}
\end{listing}

\subsubsection{Approssimazione}

Questa euristica permette di trovare una soluzione $\log(n)$-approssimata,
ma è possibile dimostrare che il fattore di approssimazione può essere
ulteriormente abbassato fino ad avere una soluzione $2$-approssimata.

TODO: aggiungere grafici e tabelle

\subsubsection{Osservazioni}

---
